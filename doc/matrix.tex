\documentclass[12pt,a4paper]{article}
%\usepackage{epsf,epic,eepic,eepicemu}
%\documentstyle[epsf,epic,eepic,eepicemu]{article}

\usepackage[pdftex]{graphicx}
\usepackage[utf8]{inputenc} %kodovani znaku v textovem souboru
%\usepackage[T1]{fontenc} %kodovani znaku na vystupu
\usepackage[czech]{babel} %prizpusobeni jazyku, napr. deleni slov
%\usepackage{a4wide}

\begin{document}
\title{Semestrální projekt MI-PAP, MI-PRC 2014/2015\\
Násobení matic \\
\vspace{10px}}
\author{Karel Fiala \\
\vspace{10px} \\
\small České vysoké učení technické v Praze\\
\small Fakulta informačních technologií\\
\small Thákurova 9, 160 00 Praha 6\\
\small Česká republika \\
\vspace{10px} \\
}
\date{\today}
\maketitle

%\oddsidemargin=-5mm \evensidemargin=-5mm \marginparwidth=.08in
%\marginparsep=.01in \marginparpush=5pt \topmargin=-15mm
%\headheight=12pt \headsep=25pt \footheight=12pt \footskip=30pt
%\textheight=25cm \textwidth=17cm \columnsep=2mm \columnseprule=1pt
%\parindent=15pt\parskip=2pt

% ===== ZADANI =====

%Kapitola 1
%- Definici problému
%- Popis sekvenčního algoritmu a jeho implementace
%
%Kapitola 2 (pro OpenMP)
%- Popis případných úprav algoritmu a jeho implementace, včetně volby datových struktur
%- Zda byla využita vektorizace (popř. proč jí nemožno využít)
%- Popis optimalizací pro dosažení lineárního zrychlení
%- Tabulkově a případně graficky zpracované naměřené hodnoty časové složitosti měřených instancí běhu (optimalizované implementace) programu s popisem instancí dat
%- Analýza a hodnocení vlastností dané implementace programu.
%
%Kapitola 3 (pro ISPC)
%- Popis případných úprav algoritmu a jeho implementace, včetně volby datových struktur
%- Zda byla využita vektorizace (popř. proč jí nemožno využít)
%- Popis optimalizací pro dosažení lineárního zrychlení
%- Tabulkově a případně graficky zpracované naměřené hodnoty časové složitosti měřených instancí běhu (optimalizované implementace) programu s popisem instancí dat
%- Analýza a hodnocení vlastností dané implementace programu.
%
%Kapitola 4 (pro CUDA)
%- Popis případných úprav algoritmu a jeho implementace, včetně volby datových struktur
%- Popis optimalizací pro dosažení efektivní implementace
%- Tabulkově a případně graficky zpracované naměřené hodnoty časové složitosti měřených instancí běhu (optimalizované implementace) programu s popisem instancí dat
%- Analýza a hodnocení vlastností dané implementace programu.
%
%Kapitola 5
%- Závěr (včetně porovnání výkonnosti všech tří verzí)
%- (případně) Literatura

\clearpage
\tableofcontents
\clearpage

\section{Definice problému a popis sekvenčního algoritmu}
\subsection{Definice problému}
\subsection{Popis sekvenčního algoritmu a jeho implementace}


\section{OpenMP}
\subsection{Popis případných úprav algoritmu a jeho implementace, včetně volby datových struktur}
\subsection{Zda byla využita vektorizace (popř. proč jí nemožno využít)}
\subsection{Popis optimalizací pro dosažení lineárního zrychlení}
\subsection{Tabulkově a případně graficky zpracované naměřené hodnoty časové složitosti měřených instancí běhu (optimalizované implementace) programu s popisem instancí dat}
\subsection{Analýza a hodnocení vlastností dané implementace programu}


\section{ISPC}
\subsection{Popis případných úprav algoritmu a jeho implementace, včetně volby datových struktur}
\subsection{Zda byla využita vektorizace (popř. proč jí nemožno využít)}
\subsection{Popis optimalizací pro dosažení lineárního zrychlení}
\subsection{Tabulkově a případně graficky zpracované naměřené hodnoty časové složitosti měřených instancí běhu (optimalizované implementace) programu s popisem instancí dat}
\subsection{Analýza a hodnocení vlastností dané implementace programu}

\section{CUDA}
\subsection{Popis případných úprav algoritmu a jeho implementace, včetně volby datových struktur}
\subsection{Popis optimalizací pro dosažení efektivní implementace}
\subsection{Tabulkově a případně graficky zpracované naměřené hodnoty časové složitosti měřených instancí běhu (optimalizované implementace) programu s popisem instancí dat}
\subsection{Analýza a hodnocení vlastností dané implementace programu}

\section{Závěr}
\subsection{Porovnání výkonnosti všech tří verzí}
\subsection{Literatura}



\end{document}
